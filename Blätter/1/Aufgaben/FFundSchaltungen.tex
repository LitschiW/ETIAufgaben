\newpage
\aufgabenbereich{Flipflops und Schaltungen}
\subsection{Volladdierer}
Zeichnen sie einen Halbaddierer auf Gatterebene. Setzen Sie dann 2 Halbaddierer (gekennzeichnent als \tikz[baseline=-0.8ex]{\node[rectangle,draw,black] at (0,0) {HA} } ) zu einem Volladdierer zusammen:\\[0.3cm]
\makeanswerbox{8}\\[1cm]
Füllen Sie die Funktionstabelle für ein Volladdierer aus:
\begin{center}
	\begin{tabular}{|c|c|c !{\vrule width 1.5pt} c|c|}\hline
		$A$&$B$&$C_{in}$&$S$&$C_{out}$	\\\hline
		0&0&0&&\\\hline
		0&0&1&&\\\hline
		0&1&0&&\\\hline
		0&1&1&&\\\hline
		1&0&0&&\\\hline
		1&0&1&&\\\hline
		1&1&0&&\\\hline
		1&1&1&&\\\hline
	\end{tabular}
\end{center}
\vspace{1cm}


\subsection{D-FlipFlop}
Was ist der Unterschied zwischen einem D-Latch und einem D-Flip-Flop?\\
\makeanswerbox{4}\\[1cm]
Füllen Sie die Funktionstabelle für ein D-Flip-Flop aus:
\begin{center}
 	\begin{tabular}{|c|c !{\vrule width 1.5pt} c|}\hline
 		$C$&$D_t$&$D_{t+1}$	\\\hline
 		&&\\\hline
 		&&\\\hline
 		&&\\\hline
 	\end{tabular}
\end{center}
(Hinweis: \texttiming{[-,timing/slope=0]HL} bezeichnet eine sinkende, \texttiming{[-,timing/slope=0]LH} eine steigende Flanke)\\[0.3cm]
Zeichnen Sie ein taktgesteuertes D-Latch auf Gatter Ebene. Makieren sie das enthaltene RS-Flipflop:\\
\makeanswerbox{7}
\newpage\noindent
Zeichnen Sie ein taktgesteuertes D-Flip-Flop. Nutzen Sie D-Latches als vorhandene Bauteile:\\
\makeanswerbox{7}

\subsection{Schieberegister}
Zeichnen Sie ein 3-Bit-Links-Schieberegister:\\
\makeanswerbox{5}\\[0.3cm]
Wofür können Schieberegister eingesetzt werden?\\
\makeanswerbox{5}
Wo kann man Schieberegister auf einem Mikrocontroller finden?\\
\makeanswerbox{4}
\aufgabenbereich{Festkommaarithmetik}
\subsection{}
\newcolumntype{Y}{>{\centering\arraybackslash}X}
Füllen Sie die Tabelle aus:
\begin{center}
	\begin{tabularx}{\textwidth}{|Y|Y|Y|Y|}\hline
		Dezimalzahl & Vorzeichenbehaftete Binärdarstellung & B-Komplement Darstellung&Hexadecimal Darstellung\\\toprule \bottomrule
		42&&&2A\\\hline
		-8&&&\\\hline
		&00110000&&\\\hline
		&10001101&&\\\hline
		&&00011010&\\\hline
		&&11111111&\\\hline
	\end{tabularx}
\end{center}
(Hinweis: $A_{(16)}=10_{(10)},B_{(16)}=11_{(10)},C_{(16)}=12_{(10)},D_{(16)}=13_{(10)},E_{(16)}=14_{(10)},F_{(16)}=15_{(10)}$)\\
\subsection{}
Konvertieren Sie $93,625_{(10)}$ jeweils in die Binär- und Hexadezimaldarstellung:\\[0.3cm]
\makeanswerbox{10}
(Hinweis: Ihr Ergebnis sollte mehr als 8 Binärstellen enthalten. Das ist in diesem Fall gewollt, Sie müssen nicht kürzen/runden.)
\newpage
\subsection{}
\noindent
Konvertieren Sie $13_{(10)}$ in die Binärdarstellung und rechen Sie $93_{(10)} - 13_{(10)}$ mittels binärer Subtraktion:\\[0.3cm]
\makeanswerbox{10}
\newpage
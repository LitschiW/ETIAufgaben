\aufgabenbereich{Fließkommaarithmetik}
Für diesen Aufgabenbereich nutzen wir Minifloats nach dem IEEE 754 Standard.\\
D.h. wir benutzen eine 8 Bit Darstellung mit einem Vorzeichen-,  3 Manitssen- und 4 Exponentbits.\\
Zahlen in Fließkommaschreibweiße werden mit $_{(2F)}$ markiert.

\subsection{Allgemeine Fragen}
Allgemeine Fragen zum Thema Fließkommazahlen.
\subsubsection{Bias}
Wie groß ist der Bias unserer Darstellung?\makeinlineanswerbox{1}{6cm}\\[0.3cm]
Was ist der Bias für eine Fließkommzahl\\[-0.2cm]mit einem Exponent der Länge 6? \makeinlineanswerbox{1}{6cm}\\
\subsubsection{Bestandteile}
Bestimmen sie die einzelnen Bestandteile der Fließkommazahl $11011100_{(2F)}$:\\
\hspace*{7.9cm}$
	\arraycolsep=1.4pt\def\arraystretch{2.2}
	\begin{array}{r l}
		V = & \makeinlineanswerbox{4}{0cm}\\
		E = & \makeinlineanswerbox{4}{0cm}\\
		M = & \makeinlineanswerbox{4}{0cm}\\
	\end{array}
$
\subsubsection{Sonderfälle}
Konvertieren Sie diese Sonderfälle in Fließkommadarstellung:\\
\hspace*{4.1cm}$
	\arraycolsep=1.4pt\def\arraystretch{2.2}
	\begin{array}{r l}
		\infty = & \makeinlineanswerbox{4}{0cm}\\
		0 = & \makeinlineanswerbox{4}{0cm}\\
		NaN = & \makeinlineanswerbox{4}{0cm}\\
		\text{kl. denormalisierte Zahl} = & \makeinlineanswerbox{4}{0cm}
	\end{array}
$\\
Woran erkennt man denormalisierte Zahlen? \makeinlineanswerbox{8}{0.34cm}
\subsection{Konvertieren}
Konvertieren Sie $11011100_{(2F)}$ in eine Dezimalzahl:\\
\makeanswerbox{5}
\subsection{Addieren}
Addieren Sie $01011100_{(2F)}$ und $01001000_{(2F)}$ mittels Fließkommaarithmetik:\\
\makeanswerbox{7}
\subsection{Multiplizieren}
Multiplizieren Sie $00011100_{(2F)}$ und $00101010_{(2F)}$ mittels Fließkommaarithmetik:\\
\makeanswerbox{7}
\subsection{n+1 Beweis}
Stellen sie $1_{(10)}$ in Fließkommaschreibweise da:\\
\makeanswerbox{1}\\[0.3cm]
Zeigen Sie anhand eines Beispiels, dass man durch das kontinuierliche Addieren von $1_{10}$ auf eine beliebige Fließkommazahl F ($\neq \infty$) niemals $\infty$ erreicht.\\
\makeanswerbox{8}
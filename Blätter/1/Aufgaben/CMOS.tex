\aufgabenbereich{Logik und CMOS-Komplexgatter}
In diesem Bereiche beschäftigen wir uns mit Logik und CMOS Komplexgattern. Es wird erwartet, dass Sie entsprechen Pull-up und Pull-down Netzwerke zeichnen.\\
Weiterhin können Sie alternativ + statt $\lor$ und $\cdot$ statt $\land$ benutzen, aber halten Sie Ihre Notation einheitlich.
\subsection{Logische Funktionen}
Füllen Sie Folgende Wahrheitstabellen aus:\\[0.3cm]
\renewcommand{\arraystretch}{1.3}
\begin{minipage}[l]{0.5\textwidth}
	\subsubsection{}
	\begin{center}
		\begin{tabular}{|c|c !{\vrule width 1.5pt} c|}\hline
			$A$&$B$& $A\lor B$\\\hline
			0&0&\\\hline
			0&1&\\\hline
			1&0&\\\hline	
			1&1&\\\hline
		\end{tabular}
	\end{center}
\end{minipage}
\begin{minipage}[r]{0.5\textwidth}
	\subsubsection{}
	\begin{center}
		\begin{tabular}{|c|c !{\vrule width 1.5pt} c|}\hline
			$A$&$B$& $A \Rightarrow B$\\\hline
			0&0&\\\hline
			0&1&\\\hline
			1&0&\\\hline	
			1&1&\\\hline
		\end{tabular}
	\end{center}
\end{minipage}\\[1cm]
\noindent
\begin{minipage}[l]{0.5\textwidth}
	\subsubsection{}
	\begin{center}
		\begin{tabular}{|c|c !{\vrule width 1.5pt} c|}\hline
			$A$&$B$& $\overline{A \land B}$\\\hline
			0&0&\\\hline
			0&1&\\\hline
			1&0&\\\hline	
			1&1&\\\hline
		\end{tabular}
	\end{center}
\end{minipage}
\begin{minipage}[r]{0.5\textwidth}
	\subsubsection{}
	\begin{center}
		\begin{tabular}{|c|c !{\vrule width 1.5pt} c|}\hline
			$A$&$B$& $\overline{A \oplus B}$\\\hline
			0&0&\\\hline
			0&1&\\\hline
			1&0&\\\hline	
			1&1&\\\hline	
		\end{tabular}
	\end{center}
\end{minipage}\\[0.3cm]
\renewcommand{\arraystretch}{1}
(Hinweis: $ (A \Rightarrow B)\equiv (\overline{A} \lor B)$)
\newpage
\subsection{Normalformen}
Geben sei folgende Funktionstabelle der Funktion $F(A,B,C)=Q$:\\
\begin{center}
	\begin{tabular}{|c|c|c|c|}\hline
		A&B&C&Q\\\hline
		0&0&0&1\\\hline
		0&0&1&0\\\hline
		0&1&0&1\\\hline
		0&1&1&0\\\hline
		1&0&0&1\\\hline
		1&0&1&0\\\hline
		1&1&0&1\\\hline
		1&1&1&1\\\hline
	\end{tabular}
\end{center}
Welche Normalform der Funktion wäre kürzer?\\
\begin{enumerate}
	\item kKNF\hspace{0.5cm}\tikz[baseline=0.5ex]{\draw (0,0) rectangle (0.5,0.5);} \\
	\item kDNF\hspace{0.5cm}\tikz[baseline=0.5ex]{\draw (0,0) rectangle (0.5,0.5);} \\
\end{enumerate}
Geben sie die Funktion in der gewählten Normalform an:\\
\raisebox{1.3cm}{
$F(A,B,C) =$}
\parbox{0.85\textwidth}{
	\begin{minipage}[t]{0.85\textwidth}
		\makeanswerbox{3}
	\end{minipage}
}
\subsection{CMOS Technologie}
Wie viele Transistoren benötigt ein OR Komplexgatter?\makeinlineanswerbox{1}{2cm}\\[0.3cm]
Wie viele Transistoren benötigt ein NAND Komplexgatter?\makeinlineanswerbox{1}{2cm}\\[0.3cm]
Was ist der Unterschied zwischen n-Mos- und p-Mos-Transistoren?\\[0.3cm]
\makeanswerbox{2}\noindent
\subsection{Vereinfachen}
Vereinfachen Sie die Formel $G(A,B,C)\equiv \overline{\left(C \lor (\overline{C} \land A) \lor (\overline{\overline{A} \lor B})\right) \land \overline{C} } $ möglichst stark: \\
\makeanswerbox{6}\noindent
\subsection{CMOS-Komplexgatter zeichnen}
\subsubsection{}
Zeichnen Sie ein (strukturgleiches) CMOS-Komplexgatter das ihrem Ergebnis aus der Vorherigen Aufgabe entspricht:\\[0.3cm]
\makeanswerbox{10}
\subsubsection{}
Zeichnen Sie folgende Funktion strukturgleich als CMOS-Komplexgatter: $H(A,B,C)\equiv \overline{(A \lor B) \land C}$.\\
\makeanswerbox{11}\\[0.3cm]
Wie viele Transistoren würden Sie benötigen? \makeinlineanswerbox{1}{7cm}
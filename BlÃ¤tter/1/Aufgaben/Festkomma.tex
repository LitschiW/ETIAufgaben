\aufgabenbereich{Festkommaarithmetik}
\subsection{Konvertieren}
\newcolumntype{Y}{>{\centering\arraybackslash}X}
Füllen Sie die Tabelle aus:
\begin{center}
	\begin{tabularx}{\textwidth}{|Y|Y|Y|Y|}\hline
		Dezimalzahl & Vorzeichenbehaftete Binärdarstellung&Hexadecimal Code der vorz. Binärd. & B-Komplement Darstellung\\\toprule \bottomrule
		42&&2A&\\\hline
		-8&&&\\\hline
		&01000101&&\\\hline
		&10011011&&\\\hline
		&&&00011010\\\hline
		&&&11111111\\\hline
	\end{tabularx}
\end{center}
(Hinweis: $A_{(16)}=10_{(10)},B_{(16)}=11_{(10)},C_{(16)}=12_{(10)},D_{(16)}=13_{(10)},E_{(16)}=14_{(10)},F_{(16)}=15_{(10)}$,\\
\hspace*{1.8cm}``vorz. Binärd.'' = vorzeichenbehaftete Binärdarstellung)
\subsection{Konvertieren II}
Konvertieren Sie $83,625_{(10)}$ jeweils in die Binär- und Hexadezimaldarstellung (Ihr Rechenweg sollte deutlich sein):\\
\makeanswerbox{10}
(Hinweis: Ihr Ergebnis sollte mehr als 8 Binärstellen enthalten. Das ist in diesem Fall gewollt, Sie müssen nicht kürzen/runden.)
\subsection{Subtraktion}
Konvertieren Sie $13_{(10)}$ in die Binärdarstellung und rechen Sie $83_{(10)} - 13_{(10)}$ mittels binärer Addition:\\
\makeanswerbox{8}
\subsection{Multiplikation}
Konvertieren sie $0,25_{(10)}$ in die Binärdarstellung und rechnen Sie  $0,25_{(10)} \cdot 13_{(10)}$ mittels binärer Multiplikation:\\
\makeanswerbox{8}
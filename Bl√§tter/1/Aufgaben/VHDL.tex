\aufgabenbereich{VHDL}
\subsection{}
Was ist der Unterschied zwischen einer \mintinline{vhdl}{architecture} und einem \mintinline{vhdl}{process} in VHDL?\\
\makeanswerbox{3.5}\\[0.3cm]
Was ist der Unterschied zwischen einem \mintinline{vhdl}{signal} und einem \mintinline{vhdl}{Port} in VHDL und wo werden diese definiert?\\
\makeanswerbox{3.5}\\[0.3cm]
Wofür nutzt man \mintinline{vhdl}{component}'s in VHDL\\
\makeanswerbox{3.5}\\[0.3cm]
Was macht der Befehl \mintinline{vhdl}{rising_edge(clk)} in einem \mintinline{vhdl}{process}?\\
\makeanswerbox{3.5}\\[0.3cm]
\newpage
\subsection{}
Gegeben sei folgender VHDL Code:
\begin{minted}{vhdl}
process(clk,reset) is
begin
	if clk'event and clk ='1' then
		%write your code here:
		
		
		
	endif;
	if reset'event and reset='0' then
		q_o <= (others => '0');
	endif;
end process;
\end{minted}
Sie haben weiterhin einen Eingang d\_i zur Verfügung, q\_o soll einen Ausgang darstellen.
Implementieren sie die Funktion eines D-Flip-Flops!\\
Was für einen Resettyp hat dieses Flip-Flop?\\
\begin{enumerate}
	\item synchron aktiv High\hfill\tikz[baseline=0.5ex]{\draw (0,0) rectangle (0.5,0.5);}\hspace*{11.5cm}\\
	\item synchron aktiv Low\hfill\tikz[baseline=0.5ex]{\draw (0,0) rectangle (0.5,0.5);}\hspace*{11.5cm}\\
	\item asynchron aktiv High\hfill\tikz[baseline=0.5ex]{\draw (0,0) rectangle (0.5,0.5);}\hspace*{11.5cm}\\
	\item asynchron aktiv Low\hfill\tikz[baseline=0.5ex]{\draw (0,0) rectangle (0.5,0.5);}\hspace*{11.5cm}\\
\end{enumerate}
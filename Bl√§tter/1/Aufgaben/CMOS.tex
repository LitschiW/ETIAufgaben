\aufgabenbereich{Logik und CMOS-Komplexgatter}
In diesem Bereiche beschäftigen wir uns mit Logik und CMOS Komplexgattern. Es wird erwartet, dass Sie entsprechen Pull-up und Pull-down Netzwerke zeichnen.
\subsection{Logische Funktionen}
Füllen sie Folgende Wahrheitstabellen aus:\\
(Hinweiß: $ (A \Rightarrow B)\equiv (\overline{A} \lor B)$)\\[0.3cm]
\begin{minipage}[l]{0.5\textwidth}
	\begin{center}
		\begin{tabular}{|c|c !{\vrule width 1.5pt} c|}\hline
			$A$&$B$& $A\lor B$\\\hline
			0&0&\\\hline
			0&1&\\\hline
			1&0&\\\hline	
			1&1&\\\hline
		\end{tabular}
	\end{center}
\end{minipage}
\begin{minipage}[r]{0.5\textwidth}
	\begin{center}
		\begin{tabular}{|c|c !{\vrule width 1.5pt} c|}\hline
			$A$&$B$& $A \Rightarrow B$\\\hline
			0&0&\\\hline
			0&1&\\\hline
			1&0&\\\hline	
			1&1&\\\hline
		\end{tabular}
	\end{center}
\end{minipage}\\[1cm]
\noindent
\begin{minipage}[l]{0.5\textwidth}
	\begin{center}
		\begin{tabular}{|c|c !{\vrule width 1.5pt} c|}\hline
			$A$&$B$& $\overline{A \land B}$\\\hline
			0&0&\\\hline
			0&1&\\\hline
			1&0&\\\hline	
			1&1&\\\hline
		\end{tabular}
	\end{center}
\end{minipage}
\begin{minipage}[r]{0.5\textwidth}
	\begin{center}
		\begin{tabular}{|c|c !{\vrule width 1.5pt} c|}\hline
			$A$&$B$& $\overline{A \oplus B}$\\\hline
			0&0&\\\hline
			0&1&\\\hline
			1&0&\\\hline	
			1&1&\\\hline	
		\end{tabular}
	\end{center}
\end{minipage}

\subsection{Allgemeine Fragen zum Thema CMOS:}
Wie viele Transistoren benötiget ein OR Komplexgatter?\makeinlineanswerbox{1}{2cm}\\[0.3cm]
Wie viele Transistoren benötiget ein NAND Komplexgatter?\makeinlineanswerbox{1}{2cm}\\[0.3cm]
Was ist der Unterschied zwischen n-Mos- und p-Mos-Transistoren?\\
\makeanswerbox{2}
\subsection{}
Vereinfachen Sie die Formel $ \overline{\left(C \lor (\overline{C} \land A) \lor (\overline{\overline{A} \lor B})\right) \land \overline{C} } $ möglichst stark: \\
\makeanswerbox{6.5}\\[0.3cm]
Zeichnen Sie ein (strukturgleiches) CMOS-Komplexgatter das ihrem Ergebnis entspricht:\\[0.3cm]
\makeanswerbox{12}

\subsection{}
Zeichnen Sie folgende Funktion strukturgleich als CMOS-Komplexgatter: $f(x)= \overline{(A \lor B)}$.\\
\makeanswerbox{12}\\[0.3cm]
Wie viele Transistoren würden Sie benötigen? \makeinlineanswerbox{1}{7cm}
\subsection{D-Flip Flop}
Gegeben sei folgender VHDL Code:
\begin{minted}{vhdl}
process(clk,reset) is
begin
	%write your code here:



	%------------
	if reset'event and reset='0' then
		q_o <= (others => '0');
	endif;
end process;
\end{minted}
\keepTogether{
	\subsubsection{}
	Sie haben weiterhin einen Eingang d\_i zur Verfügung, q\_o soll einen Ausgang darstellen.
	Implementieren sie die Funktion eines D-Flip-Flops!\\
}
\keepTogether{
	\subsubsection{}
	Was für einen Resettyp hat dieses Flip-Flop?\\[0.3cm]
	$
	\arraycolsep=4pt\def\arraystretch{1.5}
	\begin{array}{r l l}
	1.&\text{synchron aktiv High} &\tikz[baseline=0.5ex]{\draw (0,0) rectangle (0.5,0.5);}\\
	2.&\text{synchron aktiv Low} &\tikz[baseline=0.5ex]{\draw (0,0) rectangle (0.5,0.5);}\\
	3.&\text{asynchron aktiv High} &\tikz[baseline=0.5ex]{\draw (0,0) rectangle (0.5,0.5);}\\
	4.&\text{asynchron aktiv Low} &\tikz[baseline=0.5ex]{\draw (0,0) rectangle (0.5,0.5);}\\
	\end{array}$
}
%Aufgabe kommt hier her

%------------------------------------------------------------------------------------------------------------
%Befehle:
	
%Antwortboxen können folgendermaßen erzeugt werden:
%\makeanswerbox{height}
%\makeinlineanswrebox{length}{rechtes padding}
%\makecustomanswerbox{width}{height}

%Um Zeilenumbrüche innerhalb einer Aufgabe zu vermeiden nutze:
%\keepTogether{text}

%\subsubsection{} kann benutzt werden um die Aufgabe weiterhin zu unterteilen

%Falls deine Aufgabe zusätzliche usepackages benötigen sollte: trage sie einfach in die usepackage Datei in 
%ETIPAVorschlaege\SetupData\usepackage.tex ein schreibe die Funktion hinzu. packages die nicht annotiert sind,
%werden nicht akzeptiert
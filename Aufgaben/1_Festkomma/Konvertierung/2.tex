\keeptogether{
	\subsection{}
	Konvertieren Sie $93,625_{(10)}$ jeweils in die Binär- und Hexadezimaldarstellung:\\[0.3cm]
	\makeanswerbox{10}
	(Hinweis: Ihr Ergebnis sollte mehr als 8 Binärstellen enthalten. Das ist in diesem Fall gewollt, Sie müssen nicht kürzen/runden.)
}
\keeptogether{	
	\subsection{}
	Konvertieren Sie $13_{(10)}$ in die Binärdarstellung und rechen Sie $93_{(10)} - 13_{(10)}$ mittels binärer Subtraktion:\\[0.3cm]
	\makeanswerbox{10}
}
%------------------------------------------------------------------------------------------------------------
%Befehle:
	
%Antwortboxen können folgendermaßen erzeugt werden:
%\makeanswerbox{height}
%\makeinlineanswrebox{length}{rechtes padding}
%\makecustomanswerbox{width}{height}

%Um Zeilenumbrüche innerhalb einer Aufgabe zu vermeiden nutze:
%\keeptogether{text}

%Falls deine Aufgabe zusätzliche usepackages benötigen sollte: trage sie einfach in die usepackage Datei in 
%ETIPAVorschlaege\SetupData\usepackage.tex ein schreibe die Funktion hinzu. packages die nicht annotiert sind,
%werden nicht akzeptiert
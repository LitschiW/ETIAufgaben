\keepTogether{
	\subsection{CMOS-Komplexgatter 4}
	Vervollständigen Sie folgendes CMOS-Gatter:\\
	\noindent
	\begin{tikzpicture}
	
	%box and center
	\draw[blur shadow={shadow blur steps=10},fill = white] (0,0) rectangle (\textwidth,15);
	\draw[draw, black] (3,7.5) -- (\textwidth-20,7.5) node[right]{Q};
	
	%topside
	\node[draw,circle,scale = 0.8] (vdd) at (\textwidth/2,14) {$V_{dd}$} ;
	\draw (vdd.south) to[short,-*] (vdd|-1,13) -- (3,13) -- (\textwidth-100,13);
	
	%bottom side
	\node[nmos](mos1) at (\textwidth/3,5) {} ;
	\node[nmos](mos2) at (\textwidth-\textwidth/3,5) {} ;
	\node[nmos](mos3) at (\textwidth/2,3.44) {} ;
	\node[rground] (gnd) at (\textwidth/2,1) {};
	
	\draw
	(mos1.source) -- (mos3.drain) -- (mos2.source)
	(mos3.source) -- (gnd)
	(mos1.drain) to[short,-*] (mos1|-0,7.5)
	(mos2.drain) to[short,-*] (mos2|-0,7.5)
	(mos1.gate) node[left] {$A$}
	(mos2.gate) node[left] {$\overline{B}$}
	(mos3.gate) node[left] {$C$};
	
	%this is not accually how circuit tikz schould be used, but it works just fine! 
	\end{tikzpicture}\\[.3cm]
	Welche Funktion wurde hier implementiert? \makeinlineanswerbox{9}{0cm}
}\noindent
%------------------------------------------------------------------------------------------------------------
%Befehle:
	
%Antwortboxen können folgendermaßen erzeugt werden:
%\makeanswerbox{height}
%\makeinlineanswrebox{length}{rechtes padding}
%\makecustomanswerbox{width}{height}

%Um Zeilenumbrüche innerhalb einer Aufgabe zu vermeiden nutze:
%\keepTogether{text}

%\subsubsection{} kann benutzt werden um die Aufgabe weiterhin zu unterteilen

%Falls deine Aufgabe zusätzliche usepackages benötigen sollte: trage sie einfach in die usepackage Datei in 
%ETIPAVorschlaege\SetupData\usepackage.tex ein schreibe die Funktion hinzu. packages die nicht annotiert sind,
%werden nicht akzeptiert